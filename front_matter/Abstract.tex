\thispagestyle{plain}
\begin{center}
    \Large \textbf{\uppercase{Abstract}}
\end{center}

\vspace{1\baselineskip}

\noindent
Classification is the procedure to recognize, understand, as well as group ideas and objects into
given categories. Classification techniques adopt training data patterns to predict the likelihood
that subsequent data will classify into one of the given categories. Classification techniques
utilize a variety of algorithms to classify future datasets through training data patterns. In
current society, many network attacks continue to carry out various types of attacks. This work
performs data preprocessing and uses Python with machine learning algorithms to analyze the
NSL-KDD data set. We use various machine learning methods, such as decision trees, random
forests, Naïve Bayes, KNN, Gradient Boosted Trees, and SVM to analyze the confusion matrix
and predict the accuracy. We also draw the ROC curve and the AUC area. We calculate the
ACC accuracy and make a simple assessment of the quality of different algorithms. Test results
show that through data pre-processing, machine learning algorithms be performed with
extremely high accuracy.